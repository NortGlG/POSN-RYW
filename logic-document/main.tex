% !TEX program = xelatex
\documentclass[12pt,a4paper]{article}

%% ---- LaTeX syntax utility package ---- %%
\usepackage{etoolbox}
\usepackage{amsmath}
\usepackage{amsfonts}
\usepackage{amssymb}
\usepackage{pstricks}
\usepackage{pst-barcode}
%% ---- Set up paper margin ---- %%
\usepackage[a4paper,top=1in,bottom=1in,left=1in,right=1in]{geometry}

%% ---- Set up fonts and encoding ---- %%
\usepackage{fontspec}
\usepackage{xunicode}
\usepackage{xltxtra}

% Enable line breaks for Thai text
\XeTeXlinebreaklocale "th"
\XeTeXlinebreakskip = 0pt plus 2pt minus 1pt

% Set up Thai fonts
\setmainfont[%
    ItalicFont={Laksaman-Italic.otf},%
    BoldFont={Laksaman-Bold.otf},%
    BoldItalicFont={Laksaman-BoldItalic.otf},%
    Script=Thai,%
    Scale=MatchLowercase,%
    WordSpace=1.25,%
    Mapping=tex-text,%
]{Laksaman.otf}


%% ---- Load line spacing package ---- %%
\usepackage{setspace}

% Introducing hair space 
\newrobustcmd{\hrsp}{\ifmmode\mskip1mu\else\kern0.0625em\fi}


%% ---- Set up hyperlinks and colors ---- %%
\usepackage{xcolor}
\usepackage[unicode=true]{hyperref}
\hypersetup{%
    colorlinks,%
    linkcolor={red!50!black},%
    citecolor={blue!50!black},%
    urlcolor={blue!80!black},%
}
\renewcommand\UrlFont{\normalfont}


%% ---- Documents start here ---- %%
\title{
	Logic (ตรรกศาสตร์)
}
\author{
    Pana Wanitchollakit
}
\date{
%2024
}

\begin{document}

\maketitle

\section*{ประพจน์ (proposition)}
ประโยคบอกเล่าหรือประโยคปฏิเสธซึ่งบอกได้ว่า\textbf{จริง (True)} หรือ\textbf{เท็จ (False)}
\subsection*{ตัวดำเนินการทางตรรกศาสตร์ (Logical operators)}
ให้ $p, q$ เป็นประพจน์
\begin{itemize}
    \item นิเสธ (negation): $\neg\ p$
    \item และ (and): $p\land q$
    \item หรือ (or): $p \lor q$
    \item ถ้า$\ldots$แล้ว (implication): $p \rightarrow q$
    \item ก็ต่อเมื่อ (if and only if): $p\leftrightarrow q$
\end{itemize}
\subsection*{ตารางค่าความจริง (Truth table)}
\quad ตารางที่แสดงค่าความจริงทุกความเป็นไปได้ทั้งหมดของการกำหนดค่าทุกประพจน์ โดยจะมีทั้งหมด $2^N$ แถว เมื่อ $N$ เป็นจำนวนประพจน์ทั้งหมด. \\
\rule{0pt}{4ex}
ตัวอย่างเช่นเรามี 2 ประพจน์ได้แก่ $p, q$\\
\begin{center}
    \begin{tabular}{|c|c||c|c|c|c|c|}
        \hline
        $p$ & $q$ & $\neg p$ & $p \land q$ & $p \lor q$ & $p \rightarrow q$ & $p \leftrightarrow q$ \\
        \hline
        $T$ & $T$ & $F$      & $T$         & $T$        & $T$               & $T$                   \\
        \hline
        $T$ & $F$ & $F$      & $F$         & $T$        & $F$               & $F$                   \\
        \hline
        $F$ & $T$ & $T$      & $F$         & $T$        & $T$               & $F$                   \\
        \hline
        $F$ & $F$ & $T$      & $F$         & $F$        & $T$               & $T$                   \\
        \hline
    \end{tabular}
\end{center}
\subsection*{สมมูล (equivalence, $\equiv$)}
\quad ประพจน์ใดๆ 2 ประพจน์จะสมมูลกัน \textbf{ก็ต่อเมื่อ} ทั้งสองประพจน์ให้ค่าในตารางความจริง\textbf{เหมือนกันทั้งหมด} \\
\rule{0pt}{4ex}
ตัวอย่างเช่น $p\rightarrow q \equiv \neg p \lor q$ (* ใช้ในการแก้โจทย์ที่อยู่ในรูป ถ้า$\ldots$แล้ว บ่อย)
\begin{center}
    \begin{tabular}{|c|c||c||c|c|}
        \hline
        $p$ & $q$ & $\neg p$ & $\neg p \lor q$ & $p \rightarrow q$ \\
        \hline
        $T$ & $T$ & $F$      & $T$             & $T$               \\
        \hline
        $T$ & $F$ & $F$      & $F$             & $F$               \\
        \hline
        $F$ & $T$ & $T$      & $T$             & $T$               \\
        \hline
        $F$ & $F$ & $T$      & $T$             & $T$               \\
        \hline
    \end{tabular}
\end{center}

\subsubsection*{Logical equivalences}
\quad กำหนดให้ $p,q,r$ เป็นประพจน์ และ $T, F$ หมายถึง \textbf{จริง} และ \textbf{เท็จ} ตามลำดับ. \\
\paragraph{if\ldots then, if and only if ***}
\begin{itemize}
    \item $p \rightarrow q \equiv \neg p \lor q$
    \item $p \leftrightarrow q \equiv (p \rightarrow q) \land (q \rightarrow p)$
\end{itemize}
\paragraph{contra-positive ****}
\begin{itemize}
    \item $p \rightarrow q \equiv \neg q \rightarrow \neg p$
\end{itemize}
\paragraph{Identity laws *}
\begin{itemize}
    \item $p \land T \equiv p $
    \item $p \lor F \equiv p $
\end{itemize}

\paragraph{Domination laws *}
\begin{itemize}
    \item $p \land F \equiv F$
    \item $p \lor T \equiv T$
\end{itemize}

\paragraph{Idempotent laws}
\begin{itemize}
    \item $p \land p \equiv p$
    \item $p \lor p \equiv p$
\end{itemize}

\paragraph{Double negation law}
\begin{itemize}
    \item $\neg(\neg p) \equiv p$
\end{itemize}

\paragraph{Commutative law}
\begin{itemize}
    \item $p\land q \equiv q \land p$
    \item $p \lor q \equiv q \lor p$
\end{itemize}

\paragraph{Associative laws}
\begin{itemize}
    \item $(p \land q) \land r \equiv p \land (q \land r)$
    \item $(p \lor q) \lor r \equiv p \lor (q \lor r)$
\end{itemize}

\paragraph{Distributive laws *}
\begin{itemize}
    \item $p \lor (q \land r) \equiv (p \lor q) \land (p \lor r)$
    \item $p \land (q \lor r) \equiv (p \land q) \lor (p \land r)$
\end{itemize}

\paragraph{Absorption laws ***}
\begin{itemize}
    \item $p \lor (p \land q) \equiv p$
    \item $p \land (p \lor q) \equiv p$
\end{itemize}
\paragraph{Negation laws *}
\begin{itemize}
    \item $p \land \neg p \equiv F$
    \item $p \lor \neg p \equiv T$
\end{itemize}
\paragraph{De Morgan's laws ****}
\begin{itemize}
    \item $\neg(p \land q) \equiv \neg p \lor \neg q$
    \item $\neg(p \lor q) \equiv \neg p \land \neg q$
\end{itemize}


\subsubsection*{Note (Trick?)}
สำหรับน้องที่จำสูตรไม่ได้วิธีแนะนำคือลองวาด Venn-Euler diagram โดยมองว่าประพจน์เป็นเซ็ต และ $T$ คือ space $\left( \mathcal{U} \right)$ และ $F$ คือเซตว่าง ($\phi$) และมองกระบวนการทางตรรกศาสตร์เป็นกระบวนการทางเซตแทน. \\
\begin{itemize}
    \item $\land$ มองเป็น $\cap$
    \item $\lor$ มองเป็น $\cup$
    \item $\neg p$ มองเป็น $\bar{A}$
\end{itemize}


\rule{0pt}{4ex}
Fact: กระบวนการทางเซ็ต (Union, Intersect, etc.) ถูกนิยามโดยตัวดำเนินการทางตรรกศาสตร์
ตัวอย่างเช่น
\begin{itemize}
    \item $A \cup B = \left\{x \in \mathcal{U} \lvert x \in A \lor x\in B \right\}$
    \item $A \cap B = \left\{x \in \mathcal{U} \lvert x \in A \land x\in B \right\}$
    \item $\bar{A} = \left\{x \in \mathcal{U} \lvert \neg( x \in A) \right\} = \left\{x \in \mathcal{U}| x \not\in A\right\}$
    \item $\overline{(A \cup B)} = \left\{ x \in \mathcal{U} \lvert \neg(x\in A \lor y\in B) \right\} = \left\{ x \in \mathcal{U} \lvert x\not\in A \land x\not\in B \right\}$
\end{itemize}
\hrulefill
\newpage

\section*{สัจนิรันดร์ และ ความสมเหตุสมผล (Tautology and Valid argument)}
\subsection*{สัจนิรันดร์ และ คอนทราดิกชัน (Tautology, Contradiction)}
\quad \textbf{สัจนิรันดร์ (Tautology)} คือ ประพจน์ประกอบ ที่ไม่ว่าจะระบุค่าความจริงในแต่ละประพจน์ใดๆ นั้นจะให้ค่า\textbf{เป็นจริงเสมอ}\par
\textbf{คอนทราดิกชัน (Contradiction)} เหมือนสัจนิรันดร์แต่ค่าความจริงเป็น \textbf{เท็จเสมอ}
\subsubsection*{ตัวอย่างโจทย์}
จงตรวจสอบว่าประพจน์นี้เป็นสัจนิรันดร์หรือไม่
$$
    \left[ (p \rightarrow q) \land (q \rightarrow r) \right] \rightarrow (p \rightarrow r)
$$
\rule{0pt}{35ex}
\subsection*{ความสมเหตุสมผล (Valid argument)}
\quad ประพจน์ประกอบด้วย\textbf{เหตุ}หลายประพจน์และ\textbf{ผล}หนึ่งประพจน์ จะสมเหตุสมผล ก็ต่อเมื่อ นำ\textbf{เหตุมาเชื่อมด้วย และ ทั้งหมด} แล้ว เชื่อมเหตุทั้งหมดกับผลด้วย\textbf{ถ้า...แล้ว}เป็นประพจน์ใหม่ แล้วประพจน์นั้นเป็น \textbf{สัจนิรันดร์}

\subsubsection*{ตัวอย่างโจทย์}
เหตุ
\begin{itemize}
    \item $p \rightarrow q$
    \item $q \rightarrow r$
\end{itemize}
ผล
\begin{itemize}
    \item $p \rightarrow r$
\end{itemize}
ได้ประพจน์ที่ต้องตรวจสอบว่า
$$
    \left[ (p \rightarrow q) \land (q \rightarrow r) \right] \rightarrow (p \rightarrow r)
$$
\hrulefill

\newpage
\section*{กฎการให้เหตุผล (Rule of Inference)}
มีหลายเหตุการณ์ ต้องการทราบผลของเหตุการณ์

\paragraph{Modus ponens}

\begin{flalign*}
     & p                          \\
     & p \rightarrow q            \\
     & \noindent\rule{2cm}{0.1pt} \\
     & \therefore q
\end{flalign*}

\paragraph{Modus tollens}

\begin{flalign*}
     & \neg q                     \\
     & p \rightarrow q            \\
     & \noindent\rule{2cm}{0.1pt} \\
     & \therefore \neg p
\end{flalign*}

\subsection*{ตัวอย่างโจทย์ (สอวน. คอมพิวเตอร์ 2560 ข้อที่ 29)}
เหตุ
\begin{itemize}
    \item ถ้าฝนตกแล้วน้ำท่วม
    \item ถ้าน้ำท่วมแล้วจะเกิดโรคระบาด
    \item ถ้าเกิดโรคระบาดแล้วประชาชนยากจน
    \item ประชาชนไม่ยากจน
\end{itemize}
\rule{0pt}{15ex}
\hrulefill

\newpage
\section*{ประโยคเปิด และ ตัวบ่งปริมาณ (Predicate logic and Quantifier)}
\subsection*{ประโยคเปิด (Predicate logic, Open sentence)}
\quad ประโยคเปิด $P(x)$ จะเป็นประพจน์ก็ต่อเมื่อมีการใส่ตัวแปร x เข้าไปในประโยค เช่น \\
\rule{0pt}{4ex}
$P(x)$ คือ $x$ มาเข้าค่ายอบรม สอวน. ที่ระยองวิทฯ (ประโยคนี้ยังไม่เป็นประพจน์ เพราะไม่รู้ว่า $x$ คือใคร)
\begin{itemize}
    \item $P(\text{หยู})$ - หยู มาเข้าค่ายอบรม สอวน. ที่ระยองวิทฯ (เป็นประพจน์เพราะตอบได้ว่าหยูเข้าค่ายหรือไม่เข้าค่าย)
    \item $P(\text{ทหารเรือ})$ - ทหารเรือ มาเข้าค่ายอบรม สอวน. ที่ระยองวิทฯ
\end{itemize}
\rule{0pt}{4ex}
$Q(x)$ คือ $x < 10$ (ยังตอบไม่ได้ว่าจริงหรือเท็จเพราะยังไม่รู้ค่า $x$)
\begin{itemize}
    \item $Q(1)$ - $1 < 10$ (เป็นจริง)
    \item $Q(11)$ - $11 < 10$ (เป็นเท็จ)
\end{itemize}

\subsection*{ตัวบ่งปริมาณ (Quantifier)}
\quad กำหนดให้ทุก $x$ อยู่ใน space $\mathcal{U}$

\subsubsection*{Universal Quantifier (For all $\forall$)}
$\forall_x\left[ P(x) \right]$ จะเป็น \textbf{จริงได้ในกรณีเดียว} เมื่อ $P(x)$ ในทุก $x \in \mathcal{U}$ เป็น\textbf{จริงทั้งหมด} ตัวอย่างเช่น \\
\rule{0pt}{2.5ex}

ให้ $\mathcal{U} = \{1,2,3,4,5,6\}$ และ $Q(x)$ คือ $x < 10$\\
\rule{0pt}{4ex}
$\therefore \forall_x[Q(x)] \equiv T$ (เพราะทุกเลขใน $\mathcal{U}$ น้อยกว่า $10$ หมด)

\subsubsection*{Existential Quantifier (For some $\exists$)}
$\exists_x\left[ P(x) \right]$ จะเป็น \textbf{เท็จได้ในกรณีเดียว} เมื่อ $P(x)$ ในทุก $x \in \mathcal{U}$ เป็น\textbf{เท็จทั้งหมด} ตัวอย่างเช่น \\
\rule{0pt}{2.5ex}

ให้ $\mathcal{U} = \{10,20,30,40,50\}$ และ $Q(x)$ คือ $x < 10$\\
\rule{0pt}{4ex}
$\therefore \exists_x[Q(x)] \equiv F$ (เพราะทุกเลขใน $\mathcal{U}$ มากกว่าหรือเท่ากับ $10$ หมด)

\subsubsection*{การกระจายนิเสธเข้าตัวบ่งปริมาณ (Negating Quantified expression)}
\begin{itemize}
    \item $\neg\forall_x[P(x)] \equiv \exists_x[\neg P(x)]$
    \item $\neg\exists_x[P(x)] \equiv \forall_x[\neg P(x)]$
\end{itemize}
(เปลี่ยนตัวบ่งปริมาณเป็นอีกแบบแล้วกระจายนิเสธเข้าประพจน์ได้เลย) \\
ตัวอย่างเช่น \\
$$
    \neg \forall_x[x < 10] \equiv \exists_x[\neg( x < 10)] \equiv \exists_x[x \ge 10]
$$
มองเป็นคำพูดได้ว่า ไม่ใช่ทุกเลขที่น้อยกว่า 10 มีความหมายเหมือน มีบางเลขที่มากกว่าหรือเท่ากับ 10
\end{document}
